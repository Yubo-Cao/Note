\documentclass{article}
\usepackage{amsmath}
\usepackage{physics}
\usepackage{circuitikz}

\author{Yubo Cao}
\title{Quick Summary for Circuit}
\date{\today}

\begin{document}

\maketitle
\tableofcontents
\newpage

\section{Ohm's Law}

You should be able to derive all the subsequent equation from first one.

\begin{equation}
    \begin{split}
        I & = \frac{V}{R} \\
        V & = \frac{I}{R} \\
        R & = VI
    \end{split}
    \label{eq:ohms}
\end{equation}

Where $I$ is the current, $V$ is the voltage, and $R$ is the resistance.

\section{Circuit}

You should be able to classify circuit as series, parallel, and combination. For
combination circuit, take part of circuit that are parallel/series as a electrical
device to simplify the problem.

For parallel and series circuit, you should remember those relationships by heart, as
they are not provided in formula sheet. By appling the relationships, you should be
able to solve them.

All the problem can be recurively resolved in this process. 

\subsection{Series Circuit}


\begin{figure}[htb]
    \centering
    \begin{circuitikz}
        \draw
        (0,0) to[switch] (3.236, 0)
        to[battery1] (6.472,0)
        to[R,l=$R_1$] (6.472,4) -- (0,4)
        to[R,l=$R_2$] (0,0);
    \end{circuitikz}
    \caption{Series Circuit}
    \label{fig:series}
\end{figure}

A series circuit in which there are only one way for electron to flow from positive to
negative. In such circuit, one switch is sufficent to control all the electrical
devices. However, it also means that this circuit would break immediately if any
part of the circuit is broken.

Assuming the electrical devices are labeled as \ref{fig:series} shows, $R_n$, $I_n$,
and $V_n$ are the resistance, current, and voltage of device $n$. Then:

\begin{equation}
    \begin{split}
        I_T &= I_1 = I_2 = \ldots = I_n \\
        V_T &= \sum_{i = 1}^{n} V_i\\
        R_T &= \sum_{i = 1}^{n} R_i
    \end{split}
    \label{eq:series}
\end{equation}

where $I_T$ is the total current, $V_T$ is the total voltage, and $R_T$ is the total
resistance.

\subsection{Parallel Circuit}
\begin{figure}[htb]
    \centering
    \begin{circuitikz}
        \draw
        (0, 0) to[switch] (3.236, 0)
        to[battery1] (6.472, 0) -- (6.472, 2)
        to[R, l=$R_1$] (0, 2) -- (0, 2)
        to (0, 0)
        (6.472, 2) -- (6.472, 4)
        to[R, l=$R_2$] (0, 4) -- (0, 4)
        to (0, 2)
        ;
    \end{circuitikz}
    \caption{Parallel Circuit}
    \label{fig:parallel}
\end{figure}

A series circuit is a circuit in which there are multiple ways for electron to flow
from positive to negative. In such circuit, the breakage of any part of the circuit
does not affect the whole circuit (unless, you break the battery or some part that the
entire circuit share).

Use the exact same notation, the following relationship exists:

\begin{equation}
    \begin{split}
        I_T &= I \sum_{i = 1}^{n} I_i \\
        V_T &= V_1 = V_2 = \ldots = V_n\\
        \frac{1}{R_T} &= \sum_{i = 1}^{n} \frac{1}{R_i}
    \end{split}
    \label{eq:parallel}
\end{equation}

Please remember that the reciprocal of total resistance is the sum of the reciprocal of
individual resistance. If you get some number smaller than 1 in a question asking for
resistance, check again to see if you made any silly mistake.

Because of such relationship of resistance in parallel circuit, one thing always hold
is: the resistance of parallel circuit always decrease when more electrical devices are
added to the circuit.

\section{Power \& Work}

You should be able to derive rest of them by any one of them with \ref{eq:ohms}, which
is the case during the exam: only $P=VI$ will be provided to you.

\subsection{Power}

\begin{equation}
    \begin{split}
        P &= VI \\
        P &= I^2R\\
        P &= \frac{V^2}{R}
    \end{split}
    \label{eq:power}
\end{equation}

where $P$ is the power, $I$ is the current, $V$ is the voltage, and $R$ is the resistance.

\subsection{Work}

\begin{equation}
    \begin{split}
        W &= VIt\\
        W &= \frac{V^2t}{R} \\
        W &= I^2Rt
    \end{split}
    \label{eq:work}
\end{equation}

where $W$ is the work, $V$ is the voltage, $I$ is the current, and $R$ is the resistance.


\end{document}